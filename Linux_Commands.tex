\documentclass[10pt]{article}
\usepackage[utf8]{inputenc}
\usepackage{tcolorbox}
\usepackage{titling}
\pdfpagewidth=8.5in \pdfpageheight=11in
%\topmargin -.695in \evensidemargin -.5mm \oddsidemargin -.5mm
%\setlength{\textwidth}{17.05cm} \setlength{\textheight}{23.5cm}

\title{Useful Linux Commands}
\author{Christopher Malek}
\date{}
\setlength{\droptitle}{-13em}	

\newcommand{\bt}{\begin{tcolorbox}}
\newcommand{\et}{\end{tcolorbox}}
\begin{document}
\maketitle
\section{Watch Real-Time CPU Freq.}
\bt
watch -n 0.01 ``cat /proc/cpuinfo $|$ grep -i mhz''
\et

\section{If kworker is eating CPU:}
\bt
grep . -r /sys/firmware/acpi/interrupts/\\
sudo -s\\
echo ''disable'' $>$ /sys/firmware/acpi/interrupts/gpe61 (gpe61 example)
\et

\section{Repeat a Terminal Command:}
Repeat a terminal command every 'x' amount of seconds:
\bt
while true; do (command); sleep (interval); done\\
\textbf{Example:} while true; do python sumtimes.py; sleep 5; done
\et

\section{If USB Flash Drive isn't working:}
\bt
sudo ntfsfix /dev/sdaXXX
\et

\section{rsync commands:}
Exclude .cache files and show progress:
\bt
sudo rsync -a --exclude=.cache --progress /home/chris /media/chris/Storage
\et
\noindent To sync only new files:
\bt
rsync -avzh --ignore-existing /home/chris/ /media/chris/Storage
\et

\section{ls -l}
 \bt
   ls -l
 \et
\noindent Lists 9 columns in stdout:\\
1)Permissions of the files, r = read, w = write, x = execute\\
2)Number of links of the files\\
3)Owner of the files\\
4)Group that the files are assigned to\\
5)Size of the file in bytes\\
6,7,8,9) Modification time, path

\section{Changing File Permissions}
User obtains permission to execute file:
  \bt
   chmod u+x file
  \et
\noindent Others and the group loose the write permissions to files file1 and file2
  \bt
   chmod og-w file1 file2
  \et
\noindent Everyone obtains read permissions:
  \bt
   chmod a + r
  \et
Execute this to allow bash scripts to be executed:
	\bt
	chmod -x filename
	\et
\section{Rename a File}

\bt
mv file1 file2
\et

\section{Confirmation to Remove}
To confirm before deleting all files:
\bt
rm -i *
\et

\section{Filters}
Display file contents in terminal in a variety of ways:
   \subsection{cat}
   Print all the info inside data to the terminal:
     \bt
     cat data.txt
     \et

  \subsection{less}
  Browse the contents of data.txt:
  \bt
  less data.txt
  \et

  \subsection{head}
  Print first line of data.txt from the top
  \bt
  head -n 1 data.txt
  \et

  \subsection{tail}
  Same as head but it starts from the bottom of the file:
  \bt
  tail -n 2 data.txt
  \et

  \subsection{sort}
  \bt
  sort data.txt \qquad sort contents alphabetically\\
  sort -r data.txt \quad reverse sort\\
  sort -n data.txt \quad sort numerically
  \et
  
   \subsection{grep}
   Processes a text file line by line searching for a given string of characters.
   \bt
   grep kilos data.txt \qquad Print lines containing 'kilos'\\
   grep -v kilos data.txt \quad Print each line NOT containing the string 'kilos'
   \et
  
\section{scp to Raspberry Pi}
Copy file from main to Raspberry Pi:
   \bt
   scp file.ext pi@192.168.1.12:dir/ \quad The /home/pi/dir must already exist
   \et

Copy file from Raspberry Pi to main:
   \bt
   scp pi@192.168.1.12:file.ext \quad will transfer to current directory
   \et

\section{Change Swapiness}
   \bt
   sudo (editor) /etc/sysct1.conf\\
   vm.swappiness=(VALUE)
   \et
\section{Multiple Copy}
	\bt
	echo dir1/ dir2/ dir3/ | xargs -n 1 cp filename
	\et

\section{Find a File}	
	\bt
	find . -name 'filename'
	\et
	
\section{Tar and Untar}
	\bt 
	\textbf{Create:  } tar -zcvf name.tar.gz directory-name/ \\
	\textbf{Extract: } tar -zxvf name.tar.gz
	\et
 \end{document}
